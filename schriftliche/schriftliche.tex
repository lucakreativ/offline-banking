%! author=ich
\documentclass[a4paper,12pt,ngerman]{scrreport}
\usepackage[ngerman]{babel}
\usepackage[T1]{fontenc}
\usepackage[utf8x]{inputenc}
\usepackage{lmodern}
\usepackage{multicol}
\setlength{\columnsep}{24pt}
\usepackage[a4paper,margin=2.5cm,footskip=0.5cm]{geometry}

\setlength{\footheight}{1cm}
% Die nächsten vier Felder bitte anpassen:
\newcommand{\Aufgabe}{42. Bundeswettbewerb Informatik}  % Aufgabennummer und Aufgabennamen angeben
\newcommand{\TeamId}{01320}                             % Team-ID aus dem PMS angeben
\newcommand{\Teammitglieder}{Ole Fleck}

% Kopf- und Fußzeilen
\usepackage{scrlayer-scrpage}
\pagestyle{headings}
\clearpairofpagestyles
\ihead{42. BWINF\\1. Runde}
\chead{\headmark}
\automark{chapter}
\ohead{\TeamId\\The Crew - Mission BWINF}
\cfoot*{\\\thepage}

% Für mathematische Befehle und Symbole
\usepackage{amsmath}
\usepackage{amssymb}
\usepackage{amsfonts}
\usepackage{amsthm}


% Für Bilder
\usepackage{graphicx}

% Für Algorithmen
\usepackage{algpseudocode}

% Für Quelltext
\usepackage{listings}
\usepackage{color}\definecolor{mygreen}{rgb}{0,0.6,0}
\definecolor{mygray}{rgb}{0.5,0.5,0.5}
\definecolor{mymauve}{rgb}{0.58,0,0.82}
\lstset{
    keywordstyle=\color{blue},commentstyle=\color{mygreen},
    stringstyle=\color{mymauve},rulecolor=\color{black},
    basicstyle=\footnotesize\ttfamily,numberstyle=\tiny\color{mygray},
    captionpos=b, % sets the caption-position to bottom
    keepspaces=true, % keeps spaces in text
    numbers=left, numbersep=5pt, showspaces=false,showstringspaces=false,
    showtabs=false, stepnumber=1, tabsize=2, title=\lstname,
    literate=%
    {Ö}{{\"O}}1
    {Ä}{{\"A}}1
    {Ü}{{\"U}}1
    {ß}{{\ss}}1
    {ü}{{\"u}}1
    {ä}{{\"a}}1
    {ö}{{\"o}}1
}
\usepackage{listingsutf8}


% Diese beiden Pakete müssen zuletzt geladen werden
%\usepackage{hyperref} % Anklickbare Links im Dokument
\usepackage{cleveref}

% Daten für die Titelseite
\title{Offline-Banking\\
    \vspace{6mm}
    \large{!!Untertitel}}
\author{\Large{\textsc{Luca Eckenfels}}\\\textsc{Ole Fleck}\\
    \vspace{5cm}\\
\date{\today\\\vspace{12mm}
Jugend Forscht 2024: Mathe-Informatik\\\vspace{6mm}
    Erarbeitet am Schiller-Gymnasium Offenburg\\
Betreuende Lehrkraft: \\\textsc{Marek Czernohous}}
}

\begin{document}
    % INCLUDE CODE:
    % \lstinputlisting[breaklines=true, language=PYTHON, title=createKeys Programm]{FILE.py}


    \maketitle
    \tableofcontents
    \newpage
    \section*{Fachliche Kurzfassung}
    % Projektzusammenfassung für Jury-Mitglieder
    % wiss. Erkenntnisse stehen im Vordergrund

    \chapter{Ideenfindung}\label{ch:einleitung}
    \section{Motivation}
    % Welches Problem grundsätzlich lösen, wie drauf gekommen
    Schon seit geraumer Zeit ist in Deutschland das Bargeld auf dem Rückzug.
    Teilweise, insbesondere an automatischen Kassen oder Automaten, ist Kartenzahlung sogar das einzig mögliche Zahlungsmittel.
    Auch uns ist eine entsprechende Entwicklung aufgefallen, denn seit dem Besitz einer eigenen Geldkarte überwiegt die Bequemlichkeit, kein Bargeld mit herumtragen zu müssen.
    Noch angenehmer gestaltet sich die Kaufabwicklung über entsprechende Handyapps.
    Problematisch erscheint jedoch, dass für Kartenzahlung immer eine Internetverbindung des Kartenterminals benötigt wird, um Daten mit dem Bankserver auszutauschen.
    Ursprünglich erfahren haben wir von dieser Problematik an einem Ort in Deutschland ohne zuverlässige Internetverbindung, doch schnell fiel uns auf, dass dies weniger entwickelte Regionen im globalen Süden stärker betreffen sollte.
    Eine kurze Recherche ergab, dass unsere Bedenken gerechtfertigt waren:
    Während in

    \section{Anforderungen}
    % genaue Anforderungen an das fertige Programm

    \section{Problemanalyse}
    % Welche Probleme ergeben sich aus den Anforderungen

    \section{Lösungsidee}
    % Wie lösen wir diese

    \section{wissenschaftlicher Hintergrund}
    % Was weiß die Wissenschaft dazu schon


    \chapter{Umsetzung}\label{ch:umsetzung}
    \section{Planung}
    % Wie haben wir geplant

    \section{Entwicklung}
    % Wie / was haben wir so entwickelt
    \subsection{Server}
    % Wie funktioniert der Server
    \subsubsection{Datenbank}
    % und wie die Datenbank
    \subsection{Client}
    % und wie die clients
    \section{Optimierung}
    % eventuelle Optimierungen/Vebesserungen über Zeit


    \chapter{Ergebnisse}
    % Was konnten wir tatsächlich erreichen

    \section{Fazit + Ausblick}
    % Wie geht's weiter

    \chapter{Quellen}
    % Quellen
    https://de.mobiletransaction.org/beste-kartenterminals/ (06.01.24)
    https://www.nperf.com/de/map/ (06.01.24)

    \chapter{Danksagung und Unterstützungsleistungen}
    Wir haben für dieses Projekt keine fachlichen Unterstützungsleistungen in Anspruch genommen.\\
    Bedanken möchten wir uns bei:
    \begin{itemize}
        \item Marek Czernohous, unserem (früheren) Informatiklehrer, für die Begeisterung am Programmieren sowie
        \item unseren Familien für die Unterstützung
    \end{itemize}


\end{document}