\chapter{Ideenfindung}\label{ch:einleitung}
    \section{Motivation}
    % Welches Problem grundsätzlich lösen, wie drauf gekommen
    Schon seit geraumer Zeit ist in Deutschland das Bargeld auf dem Rückzug.
    Teilweise, insbesondere an automatischen Kassen oder Automaten, ist Kartenzahlung sogar das einzig mögliche Zahlungsmittel.
    Auch uns ist eine entsprechende Entwicklung aufgefallen, denn seit dem Besitz einer eigenen Geldkarte überwiegt die Bequemlichkeit, kein Bargeld mit herumtragen zu müssen.
    Noch angenehmer gestaltet sich die Kaufabwicklung über entsprechende Handyapps.
    Problematisch erscheint jedoch, dass für Kartenzahlung immer eine Internetverbindung des Kartenterminals benötigt wird, um Daten mit dem Bankserver auszutauschen.
    Ursprünglich erfahren haben wir von dieser Problematik an einem Ort in Deutschland ohne zuverlässige Internetverbindung, doch schnell fiel uns auf, dass dies weniger entwickelte Regionen im globalen Süden stärker betreffen sollte.
    Eine kurze Recherche ergab, dass unsere Bedenken gerechtfertigt waren:
    Während in

    \section{Anforderungen}
    % genaue Anforderungen an das fertige Programm
    Das Protokoll muss bestimmte Anforderungen erfüllen, um realistische Chancen für die alltägliche Anwendung zu haben:
    \begin{itemize}
        \item Transaktionen müssen ohne Internetverbindung möglich sein
        \item Transaktionen müssen sicher erfolgen, d.h.:
        \begin{itemize}
            \item Der Zahlende muss der Transaktion zustimmen.
            \item Gewährleistung, dass die Transaktion nicht mehrfach ausgeführt werden kann.
            \item Sicherheit vor Ausführung der Transaktion durch Dritte.
            \item Der Zahlende muss die Sicherheit haben, dass die Transaktionen nicht manipuliert werden können.
            
            \item Gewährleistung für den Empfänger, dass die Transaktion nicht rückgängig gemacht werden kann.
            \item Sicherheit vor Ausführung der Transaktion durch Dritte für den Empfänger.
        \end{itemize}

        \item Transaktionen müssen zu jeder Zeit durchführbar sein.
        \item Die Ausführung von Transaktionen muss in wenigen Sekunden und Schritten erfolgen.
    \end{itemize}